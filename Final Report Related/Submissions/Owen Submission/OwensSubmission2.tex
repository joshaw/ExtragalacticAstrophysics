\documentclass[pdf,color]{UoBnote}

\author{Owen McConnell}

\shorttitle{Finding Redshift Limits}
\title{Cosmic Reionization}
\date{\today}
\issue{1}

\begin{document}

%\maketitle
%\tableofcontents
%\vspace{1cm}\hrule \vspace{1cm}
%\newpage







\section{Calculating the timescale of Reionization}
In the "dark ages" of the universe, when the universe had cooled, and matter had recombined to form neutral hydrogen, the temperature was not high enough to excite either hydrogen or helium out of the ground state, and consequently neither cooled effectively via atomic line emission. Thus no photon had the sufficient energy to ionize neutral hydrogen. \\
\newline
As stars began forming, the internal temperature of the gas was high enough for atomic line emission to occur, and ionizing photons were produced. At a particular redshift the star formation rate density was be great enough to overcome the recombination rate. This point is corresponds to the critical star formation rate density. The star formation rate decreases with redshift, and conversely, the critical star formation rate increases, due to its proportionality with the clumping factor. (SEE LEWIS' SECTION)












\subsection{Critical SFRD}
The critical star formation rate density was found by an analytical model constructed by [1]

\begin{equation}
\rho^*_{SFR} = 0.027 \times f^{-1}_{esc} \left (\frac{C}{30} \right ) \left (\frac{1+z}{7} \right )^3 \left (\frac{\Omega_b h^2}{0.0465} \right )^2 (M_\odot yr^{-1} Mpc^{-3})
\end{equation}
Where $f_{esc}$ is the fractional escape of photons, z is the redshift, $\Omega_b$ is the baryonic mass density and C is the clumping factor. DISCUSSED IN LEWIS' SECTION The graph of $\rho_{SFR}$ was plotted see Figure (X)\\
\newline
GRAPH 1

%%%%%%%%%%%%%%%%%%%%%%%%%%%%%%%%%%%%%%%%%%%%%%%%%%%%%%%%%%%%%%%%%%%%%%%%%%%%%%%%%%%%
%%%%%%%%%%%%%%%%%%%%%%%%%  -----------------------------  %%%%%%%%%%%%%%%%%%%%%%%%%%
%%%%%%%%%%%%%%%%%%%%%%%%%  INSERT PICTURE OF P_CRIT HERE  %%%%%%%%%%%%%%%%%%%%%%%%%%
%%%%%%%%%%%%%%%%%%%%%%%%%  -----------------------------  %%%%%%%%%%%%%%%%%%%%%%%%%%
%%%%%%%%%%%%%%%%%%%%%%%%%%%%%%%%%%%%%%%%%%%%%%%%%%%%%%%%%%%%%%%%%%%%%%%%%%%%%%%%%%%%

\subsection{Star Formation Rate Density}
The star formation rate density was approximated from the luminosity density, as per equation (X) [2]
\begin{equation}
\rho_{SFR}(M_\odot yr^{-1} Mpc^{-3})=1e^{-28} \rho_L 
\end{equation}
where $\rho_L$ is spectral luminosity density in the region of $2150\pm 650 \AA$ in units of $(ergs \ s^{-1} Hz^{-1} Mpc^{-3})$. Within this region it was assumed that the spectral luminosity density is constant. It was calculated from the integral of the Schechter function, multiplied by the luminosity (See Figure X.X), in terms of spectral luminosities. Since the characteristic magnitude parameters of the Schechter function were obtained in terms of AB Magnitudes, a conversion between AB Magnitudes and Spectral Luminosities was used. SEE NEXT 2 FORMULAE
\begin{equation}
F_v = 10^{\displaystyle\left (\frac{5-AB-48.6}{2.5}\right )} 
\end{equation}
Where $F_v$ is the Spectral Flux Density in units of $(ergs \ s^{-1} Hz^{-1} cm^{-2})$ and AB is a monochromatic AB Magnitude. SEE APPENDIX X - LEWIS. This was used in conjunction with
\begin{equation}
L_v=4\pi d^2 F_v
\end{equation} 
Where $L_v$ is a Spectral Luminosity in units of $ergs \ s^{-1}$, and d is 10 parsecs in units of cm. \\
\newline
The lower luminosity limit on the Schechter function in figure (X.X) corresponds to the spectral luminosity of galaxies with the smallest mass possible, discussed in NEXT SECTION
\subsubsection{Minimum Galaxy Mass}
In order to find the minimum mass, the the redshift at which the Jeans mass(equation (X)) and the characteristic mass of a dark matter halo, (equation(X)), were equal, was considered.[3]

\begin{equation}
M^*=10^{13}h^{-1}(1+z)^{-4} M_\odot
\end{equation}

 The Jean's Mass defines the critical mass of a cloud before it can collapse to form a galaxy, thereby limiting the minimum mass of a star forming galaxy.[3]
\begin{equation}
M_J\approx 5.73\times10^3 \left (\frac{1+z}{10} \right )^{3/2} M_{\odot}
\end{equation}

At the redshift at which these equations are equal, the mass of the galaxy was assumed to break away from the rest of the matter of the universe, and was considered separate and distinct from it, for all calculations. This assumption ensured that this was a minimum mass of a galaxy, as no additional matter could contribute to the mass of the gas after this. The redshift at which this occurs was calculated to be z=106. This corresponds to a minimum mass of a galaxy of $M=1.1\times10^5M_\odot$.\\
\newline
Adopting a Mass-Luminosity ratio of 1 for simplicity, the minimum mass was found to correspond to an upper limit on the luminosity of the galaxy of $1.1\times 10^5 L_\odot$. However this was a bolometric luminosity, and not a spectral luminosity i.e. it represented power over all frequencies, as opposed to a distinct frequency. Thus, a conversion between bolometric luminosity and spectral luminosity was formulated. SEE NEXT SECTION

\subsection{Bolometric Luminosity to Spectral Luminosity}
In order to deduce a relation between bolometric luminosity and spectral luminosity, it was necessary to approximate the shape of the spectrum of a generic galaxy. A blackbody spectrum of temperature $1\times 10^5$K. was used to represent this spectrum, as this temperature corresponds to NEED TO FILL THIS IN \\
\newline
As spectral luminosity is luminosity per Hertz, it can be written that
\begin{equation}
\int^{\infty}_{-\infty}L_v dv = \int^{\infty}_{0}L_v dv = L_{Bol}
\end{equation} 
Where $L_{Bol}$ is the bolometric luminosity, in units of $ergs \ s^{-1}$
The Intensity of a blackbody is proportional to the spectral luminosity, at a constant solid angle and surface area, therefore it follows 
\begin{eqnarray}
L_v= kI(v,T) \\
\int^{\infty}_{0}L_v dv=L_{bol} \\
k\int^{\infty}_{0}I dv=L_{bol} \\
L_v = \frac{I(v,T)}{\int^{\infty}_{0}I dv} \times L_{bol}
\end{eqnarray}
where k is a constant, and $I(v,T)$ is the Intensity as a function of frequency and temperature in units of $ergs \ s^{-1} \Omega^{-1} A^{-1} Hz^{-1}$. $L_{Bol}/k$ was computed to be 75224.6 $ergs \ s^{-1} \Omega^{-1} A^{-1}$ The Intensity of a blackbody against frequency is as shown in figure X.
\begin{equation}
I(v,T)=\frac{8\pi v^2}{c^3}\frac{hv}{e^\frac{hv}{kT}-1}
\end{equation}
Where, within $I(v,T)$, v is the frequency corresponding to $2150\pm 650 \AA$, and T=$1.1\times 10^5K$. The lower limit of spectral luminosity in the UV range, therefore, was calculated to be $9.755\times 10^{21} ergs \ s^{-1} Hz^{-1}$, for a frequency corresponding to 2150$\AA$.
\clearpage
\subsection{Higher Redshift Limit}
In order to compute the Luminosity density - it was necessary to convert the Schechter Function Integral into a Gamma function. See APPENDIX X
This permitted the actual star formation rate density to be computed, and the graph of star formation rate density against critical star formation rate density was plotted See Figure (X.X)

GRAPH2\\
\newline
The resultant redshift was found to be 17.82.

\subsection{Lower Limit of Redshift}
Using estimates of the fractional escape and zeta, we can use the formula (X.X) combined with formula (X.X) to calculate the rate of reionization. From this, and the number of neutral hydrogen particles in the universe, it is possible to calculate the timescale of reionization. 

\begin{equation}
t =\frac{dnion}{dt}\times \frac{1}{n_H}
\end{equation}
Where t is the time between the start of reionization and the end, in seconds, nion is the number of ionizing photons produced and $n_H$ is the number of Hydrogen atoms in the universe. \\
\newline
As the number of photons produced roughly equals the number of neutral hydrogen in the universe, it is possible to calculate a limit on the redshift associated with the end of reionization, given the corresponding opposite limit. \\
\newline
A simulation of this was undergone using the equation above, and the resultant timescale was found to be NEED TO GIVE FINAL RESULT

\section{Appendix 1 - gamma function conversion}

\begin{eqnarray}
\rho_L = \int^{\infty}_{L'=L} \phi(L')L'dL'=\ \ \ \ \ \ \ \frac{\phi^*}{L^*}\int^{\infty}_{L'=L}\left (\frac{L'}{L^*} \right )^{\alpha}e^\frac{-L'}{L^*}L'dL'\\
= \ \ \ \frac{\phi^*}{L^*}\int^{\infty}_{L'=L}\left (\frac{L'}{L^*}\right )^{\alpha}e^\frac{-L'}{L^*}\frac{L'L^*}{L^*}dL'\\
= \ \frac {\phi^*}{L^*}\int^{\infty}_{L'=L}\left ( \frac{L'}{L^*} \right )^{\alpha+1}e^\frac{-L'}{L^*}L^{*2}d\frac{L'}{L^*} \\
= \ \ \ \phi^*L^*\int^{\infty}_{L'=L}\left ( \frac{L'}{L^*} \right )^{\alpha+1}e^\frac{-L'}{L^*}d\frac{L'}{L^*} \\
= \ \ \ \ \ \ \ \ \ \ \ \ \ \ \ \ \ \ \phi^*L^*\Gamma(\alpha+2, L/L^*)
\end{eqnarray}

\section{References}

[1]Journal: Astrophysical Journal - ASTROPHYS J , vol. 731, 2011 "Keeping the Universe ionised: photoheating and the high-redshift clumping factor of the intergalactic gas" Andreas H. Pawlik \\
\newline
[2]Ann.Rev.Astron.Astrophys.36:189-231,1998 "Star Formation in Galaxies Along the Hubble Sequence" Robert C. Kennicutt, Jr \\
\newline
[3]\begin{verbatim}http://www.astro.caltech.edu/~jlc/ay219_spring2010/star_form_0z.pdf
Judith G. Cohen Star Formation At Low Metallicity\end{verbatim}



\end{document}


%and the Absolute Magnitude (i.e. the dimmest ionizing galaxy.) DOES MINIMUM MASS GIVE THIS?

%\subsection{Press Schechter Formalism}

%PSF is a method of obtaining the number of objects with a specific mass within a certain volume. It assumes that the universe linearly “clumped” at the beginning of the universe, until a point where the density of the clumps break away from the rest of the universal expansion, and is treated as a massive body which collapses rapidly. (this occurs at around $\delta_c$\~1.68 – Gunn and Gott). Press and Schechter suggest a probability distribution function of =:

%\begin{equation}
%p(M,z)=-2p_0 \frac{\Delta P [\delta_v>\delta_c(z)]} {\Delta M} dM
%\end{equation}
%Where $p_0$ is the mean density of the universe, and $\delta_c(z)$ is the overdensity threshold per redshift. P is the cumulative probability distribution of $\delta_v$ (volume)\\
%\newline
 %It is from this that Schechter functions of luminosity and magnitude could arise....

%WRITE ABOUT THE PROGRAM




%\begin{equation}
%M_J=\left ( \frac{5kT}{Gm}\right ) ^{3/2} \left ( \frac{3}{4\pi\rho} \right ) ^{1/2}
%\end{equation}
%\begin{equation}
%M_J = 5.73\times 10^3\left (\frac{\Omega_mh}{0.15} \right )^{-1/2} \left (\frac{\Omega_b h^2}{0.022}\right )^{-3/5}  \left ( \frac {1+z}{10} \right ) ^{3/2} M_\odot
%\end{equation}


%IS BOLOMETRIC -> ABSOLUTE? NO ABSOLUDE IS IN A PARTICULAR BAND