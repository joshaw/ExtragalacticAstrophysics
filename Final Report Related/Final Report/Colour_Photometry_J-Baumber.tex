%!TEX root = mainfile.tex

\section{Photometry and Colour} % (fold)
\label{sec:Photometry_Colour}
(Joe, except~\ref{ssub:sources_of_contamination} Mike)

	Photometry today is used primarily with CCDs, which can convert the transmitted flux into an electric signal which can then be interpreted as a magnitude. The magnitude system used in this project was the AB magnitude system as outlined in Section~\ref{ssub:ab_magnitude}.

	In this project, wide and intermediate filters have been used predominantly. A common method for making these filters is to use coloured glass which will either pass all light above a certain wavelength or all light up to a certain wavelength. These are known as cutoff filters. A bandpass filter can be made by combining two types of coloured glass, one which will act as the low wavelength cutoff and the other as the high wavelength cutoff. Filters don't transmit 100\% of the wavelengths that are allowed to pass, and the cutoff isn't at an exact frequency.

	To detect Lyman-break galaxies the dropout method is used, which uses at least three filters to get enough spectral information to identify the object as a candidate Lyman break galaxy (see dropout method Section~\ref{ssub:dropout_technique}). However, observing the drop due to the Gunn-Peterson effect is not enough to confirm the identities of these candidates so other observational methods need to be used to be certain the object is a Lyman break galaxy and not a contaminant. One of the most effective ways of eliminating contaminants is to use the colour of the object, obtained from the difference in magnitude between different filters. The filters used for each telescope were decided by determining where the Lyman-break wavelength would appear at a certain redshift, using equation~(\ref{eq:dropout_wavelength}) and finding which filter would contain this wavelength. The two filters either side of this filter were chosen as well for colour observations and for the dropout technique. The table below shows some example filters available in James Webb Space Telescope and Euclid for a range of wavelengths. For each set, the Lyman break would be observed in the middle filter listed. 
	\begin{table}[ht]
		\centering
			\begin{tabular}{c|c|c}
				Redshift range &Telescope &Filters   \\
				\hline \hline
				6-7.5	   &James-Webb&  F070w, F090w, F115w \\
				7.5-8.5&James-Webb&  F090w, F115w, F150w \\
				8.5-10 &Euclid&  Y, J, H\\
				10-14  &James-Webb& F115w, F150w, F200w\\
				14-15  &0.105& F150w, F200w, F277w\\
			\end{tabular}
		\caption{Table showing filters used for different redshift ranges}
		\label{tab:colour_filters}
	\end{table}.


    \subsection{Contaminants} %fold
    \label{sub:Contanimants}
    	\subsubsection{Sources of Contamination} % (fold)
    	\label{ssub:sources_of_contamination}
    	%Need to be one level lower, may change

		    \subsubsection*{Low Mass Stars} % (fold)
		    \label{sub:low_mass_stars}
		        These can easily be identified due to the high resolution imaging provided by the telescopes that will be chosen for investigation. The point-spread function (PSF) obtained will allow us to determine which sources are point-like and which are extended. We should be able to avoid significant contamination by removing any point-like sources from the results as all galaxies should have a great enough diameter to be identified as extended sources.
		    % subsection low_mass_stars (end)

		    \subsubsection*{Spurious Sources} % (fold)
		    \label{sub:spurious_sources}
		        By stipulating that we will be requiring detections in two or more bands at $S/N=5$, the influence of spurious sources will be greatly limited. Further colour techniques can be employed to remove contamination by these sources. Additionally, inspecting the negative with the same requirements for detection we are able to easily identify any such sources.\cite{Bouwens2011}.
		    % subsection spurious_sources (end)

		    \subsubsection*{Supernovae and Other Transient Sources} % (fold)
		    \label{sub:supernovae_and_other_transient_sources}
		        Events such as Supernovae happen incredibly quickly releasing a vast amount of energy, as seen in Figure~\ref{fig:SNe_1987a}. These events can spoil images due to their short duration by introducing new data in only a portion of the sample. These effects are usually only considered when taking exposures years apart or when combining multiple sources over a long time scale. Such events are very unlikely to contaminate our results as we propose to take our images close in time.
		        \begin{figure}[htbp]
		            \centering
		            \includegraphics[width=0.6\textwidth]{../Images/SNe_1987a.jpg}
		            \caption{The shock wave from Supernova 1987a imaged by HST in 2006.\label{fig:SNe_1987a}}
		        \end{figure}
   			 % subsection supernovae_and_other_transient_sources (end)

		    \subsubsection*{Lower Redshift Sources and Photometric Scattering} % (fold)
		    \label{sub:lower_redshift_sources_and_photometric_scattering}
		        This category is likely to provide the greatest source of contamination for the surveyed area. It will do so increasingly at high redshifts where its effect on the faintest magnitudes is most greatly felt. Its effect is most influential for observations with a small S/N ratio however by fixing this at a level of $S/N = 5$ in each band we can be confident that the contamination will be low. Detecting a source in another band such as b435, v606 or i775 for YJH photometry would class it as a contaminant and it should then be removed from the sample, however we won't use this technique as it would too greatly increase the observing time. Colour techniques, such as colour windows can also be used to remove these sources.
		    % subsection lower_redshift_sources_and_photometric_scattering (end)

    	% subsubsection sources_of_contamination (end)
    	\subsubsection{Eliminating Contaminants} % (fold)
    	\label{sub:eliminatiing_contaminants}
			The \emph{colour} of an object in photometry is defined as the difference in magnitude between two filters\cite{Romanishin}. If there are two filters, for the purpose of an example let them be called A and B, where A has a lower central wavelength, the colour for an object in these two filters would be
			\begin{align}
				m_A-m_B=-2.5\log\left(\frac{f_A}{f_B}\right).
			\end{align}
			where $f_A$ and $f_B$ are the specific flux denisites of the filters A and B respectively\cite{Romanishin}. The colour is therefore equivalent to the ratio of specific flux densities, this means two objects with the same colour can have different magnitudes in each of the filters. If the colour is positive it is said to be `red' and if it is negative it is said to be `blue', i.e. an object which is red between two filters has a lower flux in the blueward filter compared to the redward filter and an object which is red has a larger flux in the blueward filter than the redward one.  The larger the value of colour is, the `redder' the object is said to be. To eliminate contaminants from observations, a colour-colour diagram can be made using three filters, with two colour values for each object. For instance if the observations were done in the J, H and K filters, the colour-colour diagram would be (H-K) plotted against (J-H). Although this study does not take any observtions, colour-colour diagrams can be built up for observations by simulating a catalog of Lyman break galaxies at high redshift and determine colour windows for observations using the program Hyperz. Lyman-break galaxies are very red in colour for the filter blueward of the break and over the break due to the extreme drop in flux blueward of the Lyman-break, and they will be a low red value or even blue value for the colour between the filter over the break and redward of the break, due to the decrease in flux past the break. In this way contaminants as described in Section~\ref{ssub:sources_of_contamination} can be removed quickly from observations.
    	% subsection eliminatiing_contaminants (end)
	%subsection Eliminating_Contaminants (end)
% section contaminants (end)

    \subsection{Hyperz} %fold
	\label{sub:Hyperz}
		To eliminate contaminats and predict colour windows for observations the program Hyperz and its subprogram `make\_catalog' can be used to produce a catalog of synthetic galaxies giving their magnitudes in different filters at different redshifts.

        \subsubsection{Inputs} %fold
        \label{subsub:Hyperz_inputs}
			To produce this catalog, a set of inputs are put into the catalog file. The operation of the program is quite complex and is only summarised here. Also, the operation of make\_catalog is slightly different to Hyperz and a full manual for make\_catalog was not obtainable. The program starts with a sample Spectral Energy Distribution (SED), which has key features such as the Lyman break which will appear as the redshift is increased. The program combines this with a sample burst spectrum, which assumes that the stars formed quickly, which is a good assumption for high redshift galaxies\cite{hyperz}. The predictions group's Schechter function used the \SI{1500}{\angstrom} rest UV wavelength with the assumption that the flux from a Lyman break galaxy was approximately the same for \SI{1350}{\angstrom} to \SI{1750}{\angstrom}. The program uses a known magnitude in a reference filter to fit the SED and to find the magnitde of the object in other bands. Using the predictions group's program, a range of magnitudes can be found for a certain redshift interval by looking at the number of galaxies at a certain magnitude and redshift. As the redshifted \SI{1500}{\angstrom} line will move with redshift, a range of reference filters were required to cover the redshift range for the observing strategy. As the flux is assumed to be constant over the range \SI{1350}{\angstrom}--\SI{1750}{\angstrom}, this meant a single filter could be used for a wide interval of redshifts. The reference filters were chosen to have a central wavelength as close to the central \SI{1500}{\angstrom} wavelength for that redshift as possible. The reference magnitudes were based upon the predictions group's program for calculating the number density of galaxies. For each redshift range a lower magnitude was chosen based on whether any galaxies were observed below the chosen magnitude. The dimmest magnitude was chosen to be 35, as observing anything dimmer than this would be very challenging. The reference magnitude ranges are therefore different sizes making it more difficult to compare similar redshifts. If there had been time, further consideration of reference magnitudes would have been desirable. Below is a table listing the reference filters used. Note that filter 44 was not correctly labelled in the database of filters. This created unforseen problems for the Euclid magnitudes and it wasn't apparant for a long time that the reference filter was to blame.
			\begin{table}[ht]
				\begin{center}
					\begin{tabular}{c|c|c}
						Redshift range & Reference filter number&Reference magnitudes \\
						\hline \hline
						6-7.5	   &34&27-35\\
						7.5-8.5&55&27.5-35\\
						8.5-10 &44&28-35\\
						10-14  &35&29-35\\
						14-15  &72&30-35\\
					\end{tabular}
				\end{center}
				\caption{Table showing reference filters and magnitudes used}
				\label{tab:reference_filters}
			\end{table}.

			There are several other inputs which are easy to understand. The user chooses the range of redshifts for the catalog and the formation redshift, which is when the galaxies started forming. For consistency between the two groups, the same cosmological constants were used as the predicitons group, which were: $\Omega_M=0.27$, $\Omega_\Lambda=0.73$ and $H_0=\SI{71}{\kilo\metre\per\second\per\mega\parsec}$. Two other parameters which are slightly more complex are the age of the galaxies and the reddening law. To simplify the calculations, the age of the galaxies was set such that all galaxies at the same redshift have the same age, forming at the same moment at the formation reshift. The other choice would be for the galaxies to be born somewhere randomly between the formation redshift and the observed redshift. The reddening law is a more detailed input and is considered below.
		%subsubsection Inputs (end)

		\subsubsection{Reddening Law} % (fold)
		\label{ssub:reddening_law}
			Hyperz can account for reddening of galaxies. Reddening refers to the SED of a galaxy appearing redder than it actually is and is one of the effects resulting from the presence of dust. Dust is matter within galaxies that inteferes with the photons travelling towards the observer. Through spectral analysis, it has been determined that dust is composed of substances such as carbon, silicate materials, water and ammonia in the form of ice amongst other substances \cite{stein1983dust}. Dust can absorb (and emit) photons, with most absorbtion occuring in the UV and therefore the spectrum appears reddened\cite{stein1983dust}. Hyperz allows the user to select one of five laws to apply to the catalogs produced, and the law that was chosen for the purposes of this study was from Calzetti et al. (2000). There were two reasons for this: it seems to be the law that most recent papers use and appears to be a more general approximation than other laws which are based on specific galaxies. Furthermore it is a law for starburst galaxies, which corresponds to the sample spectrum used. The program asks for a maximum and minimum value of extinction in terms of magnitude, defined as $A_V$. Several steps are required to get to this required value. Starting with the equation
			\begin{align}
				A_\lambda=k(\lambda)E(B-V)=\frac{k(\lambda)A_V}{R_V}
			\end{align}
			where $A_\lambda$ is defined as the extinction at a certain wavelength, $k(\lambda)$ is the reddening curve, E(B-V) is the colour excess and $R_V$ is a constant \cite{hyperz}. This can be rearranged to find $A_V$ giving
			\begin{align}
				A_V=\frac{k(\lambda)A_\lambda}{R_V}.
			\end{align}
			$R_V$ is given as $4.05 {\pm} 0.8$ \cite{hyperz} for the Calzetti law and
			\begin{align}
				k(\lambda)=2.659(-1.857+\frac{1.040}{\lambda})+R_V
			\end{align}
			for $0.12{\mu}m \le \lambda \le 0.63{\mu}m$ for the Calzetti law\cite{hyperz}. $\lambda$ is assumed to be the emitted wavelength from the galaxy, which will be assumed to be 1216$\angstrom$ to simplify the calculation. The only unknown is $A_\lambda$. From the equation
            \begin{align}
				f_{obs}(\lambda)=f_{int}(\lambda)10^{-0.4A_\lambda}
			\end{align}
			where $f_{obs}$ is the observed flux and $f_{int}$ is the intrinsic flux, which is the flux if no reddening occured\cite{hyperz}, it can be seen that if $A_\lambda=0$ then there is no reddening In this case, the value of $A_V$ is zero and this can then be set for the minimum value for the reddening law. Although it will be very improbabable for there to be no extinction, it is theoretically possibly and simplifies the situation. The maximum value for $A_V$ is harder to find as this project doesn't take any observations and the program from the predictions group gives intrinsic fluxes, so values were found using NED's Coordinate and Galactic Extinction Calculator\cite{NEDex}. The Right Ascension and Declination of a set of galaxies from a redshift range of 6--9 from the paper by Lorenzoni et al. \cite{lorenzoni2013constraining} were inputted into the calculator, which then found values of $A_\lambda$ for a range of filters. For each redshift the filter chosen contained the redshifted Lyman break for those galaxies, the galaxy with the maximum extinction value was chosen for each redshift range. In the table below are the values obtained. For redshift 10--15 the data was extrapolated to find values. The maximum difference in $A_\lambda$ between redshifts was found for the known galaxies, and then this was applied for each increase in redshift.
			\begin{table}[ht]
				\begin{center}
					\begin{tabular}{c|c|c}
						Redshift range & $A_\lambda$ & $A_V$  \\
						\hline \hline
						6-7.5	   &0.013&  0.0044 \\
						7.5-8.5&0.013&  0.0044 \\
						8.5-10 &0.036&  0.0122\\
						10-14  &0.082&  0.0277\\
						14-15  &0.105&  0.0355\\
					\end{tabular}
				\end{center}
				\caption{Table showing values of extinction for different redshift values.}
				\label{tab:extinction_values}
			\end{table}.

			These values were put into the catalog. Many assumptions were made to find this value, so it may be incorrect. Articles tend to have a much bigger value for the maximum extinction, for example in the Hyperz manual it is 1.2. However the best possible figure was obtained with the resources avalaible. The extinction due to the Milky Way has not been considered which might have made a significant difference.
		% subsubsection reddening_law (end)

		\subsubsection{Vega to AB Conversions} % (fold)
		\label{ssub:vega_to_ab_conversions}
			Hyperz works in Vega magnitudes so AB conversions are needed. These conversions are complex and therefore conversions for a ground based telescope for typical filters have been used from Smith et al. (2008) \cite{Graham} Filters were compared to the nearest equivalent but some will inevitably be different from the actual values. The reference filters were from the database of filters that came with Hyperz and so came ready with conversions to AB. The conversions for the reference filters were applied for the input into the program, and the output magnitude in the various filters were changed to AB from Vega. The table below shows the conversion for each filter chosen for observation.
			\begin{table}[ht]
				\begin{center}
					\begin{tabular}{c|c}
						Filters & M(AB)-M(Vega) \\
						\hline \hline
						F070w, F090w,  & 0.5 \\
						F115w, Euclid Y, Euclid J	& 0.9\\
						 F150w, Euclid H	& 1.4\\
						F200w, F275w & 1.9\\
					\end{tabular}
				\end{center}
				\caption{AB conversions for filters}
				\label{tab:AB_conversion}
			\end{table}
		% subsubsection vega_to_ab_conversions (end)

		\subsubsection{Output} % (fold)
		\label{ssub:output}
			After running the executable file, the output is shown in a catalog, each redshift calaculated is random, so the galaxies are in a random order. From the output, the colour of an object can be found after the AB conversions have been made to the magnitudes and a colour-colour diagram can then be plotted.
		% subsubsection output (end)
	%subsection Hyperz (end)

	\subsection{Results for Colour} %fold
	\label{sub:Results_for_Colour}
		Below are the results for four of the specified ranges, the redshift range 14 to 15 was not included here as there was an unresolved problem with the F275w filter on NIRcam, which will be discussed below. The colour windows were found by finding the minimum value of the y-axis and the maximum value of the x-axis and stating that the colour window must be greater than or equal to and less than or equal to the values on the axes respectively, isolating the upper left quadrant of each colour-colour diagram to be the colour window.
		\begin{figure}[htbp]
			\begin{minipage}[c]{0.5\linewidth}
				\centering
					\begingroup\endlinechar=-1
						\resizebox{\textwidth}{!}{%
							\input{GRAPH_color_graph1.tex}
						}\endgroup
				\caption{A\label{fig:col1}}
			\end{minipage}
			\begin{minipage}[c]{0.5\linewidth}
				\centering
					\begingroup\endlinechar=-1
						\resizebox{\textwidth}{!}{%
							% GNUPLOT: LaTeX picture with Postscript
\begingroup
  \makeatletter
  \providecommand\color[2][]{%
    \GenericError{(gnuplot) \space\space\space\@spaces}{%
      Package color not loaded in conjunction with
      terminal option `colourtext'%
    }{See the gnuplot documentation for explanation.%
    }{Either use 'blacktext' in gnuplot or load the package
      color.sty in LaTeX.}%
    \renewcommand\color[2][]{}%
  }%
  \providecommand\includegraphics[2][]{%
    \GenericError{(gnuplot) \space\space\space\@spaces}{%
      Package graphicx or graphics not loaded%
    }{See the gnuplot documentation for explanation.%
    }{The gnuplot epslatex terminal needs graphicx.sty or graphics.sty.}%
    \renewcommand\includegraphics[2][]{}%
  }%
  \providecommand\rotatebox[2]{#2}%
  \@ifundefined{ifGPcolor}{%
    \newif\ifGPcolor
    \GPcolortrue
  }{}%
  \@ifundefined{ifGPblacktext}{%
    \newif\ifGPblacktext
    \GPblacktexttrue
  }{}%
  % define a \g@addto@macro without @ in the name:
  \let\gplgaddtomacro\g@addto@macro
  % define empty templates for all commands taking text:
  \gdef\gplbacktext{}%
  \gdef\gplfronttext{}%
  \makeatother
  \ifGPblacktext
    % no textcolor at all
    \def\colorrgb#1{}%
    \def\colorgray#1{}%
  \else
    % gray or color?
    \ifGPcolor
      \def\colorrgb#1{\color[rgb]{#1}}%
      \def\colorgray#1{\color[gray]{#1}}%
      \expandafter\def\csname LTw\endcsname{\color{white}}%
      \expandafter\def\csname LTb\endcsname{\color{black}}%
      \expandafter\def\csname LTa\endcsname{\color{black}}%
      \expandafter\def\csname LT0\endcsname{\color[rgb]{1,0,0}}%
      \expandafter\def\csname LT1\endcsname{\color[rgb]{0,1,0}}%
      \expandafter\def\csname LT2\endcsname{\color[rgb]{0,0,1}}%
      \expandafter\def\csname LT3\endcsname{\color[rgb]{1,0,1}}%
      \expandafter\def\csname LT4\endcsname{\color[rgb]{0,1,1}}%
      \expandafter\def\csname LT5\endcsname{\color[rgb]{1,1,0}}%
      \expandafter\def\csname LT6\endcsname{\color[rgb]{0,0,0}}%
      \expandafter\def\csname LT7\endcsname{\color[rgb]{1,0.3,0}}%
      \expandafter\def\csname LT8\endcsname{\color[rgb]{0.5,0.5,0.5}}%
    \else
      % gray
      \def\colorrgb#1{\color{black}}%
      \def\colorgray#1{\color[gray]{#1}}%
      \expandafter\def\csname LTw\endcsname{\color{white}}%
      \expandafter\def\csname LTb\endcsname{\color{black}}%
      \expandafter\def\csname LTa\endcsname{\color{black}}%
      \expandafter\def\csname LT0\endcsname{\color{black}}%
      \expandafter\def\csname LT1\endcsname{\color{black}}%
      \expandafter\def\csname LT2\endcsname{\color{black}}%
      \expandafter\def\csname LT3\endcsname{\color{black}}%
      \expandafter\def\csname LT4\endcsname{\color{black}}%
      \expandafter\def\csname LT5\endcsname{\color{black}}%
      \expandafter\def\csname LT6\endcsname{\color{black}}%
      \expandafter\def\csname LT7\endcsname{\color{black}}%
      \expandafter\def\csname LT8\endcsname{\color{black}}%
    \fi
  \fi
  \setlength{\unitlength}{0.0500bp}%
  \begin{picture}(7200.00,4320.00)%
    \gplgaddtomacro\gplbacktext{%
      \put(849,595){\makebox(0,0)[r]{\strut{} 8}}%
      \put(849,991){\makebox(0,0)[r]{\strut{} 8.5}}%
      \put(849,1387){\makebox(0,0)[r]{\strut{} 9}}%
      \put(849,1782){\makebox(0,0)[r]{\strut{} 9.5}}%
      \put(849,2178){\makebox(0,0)[r]{\strut{} 10}}%
      \put(849,2574){\makebox(0,0)[r]{\strut{} 10.5}}%
      \put(849,2970){\makebox(0,0)[r]{\strut{} 11}}%
      \put(849,3365){\makebox(0,0)[r]{\strut{} 11.5}}%
      \put(849,3761){\makebox(0,0)[r]{\strut{} 12}}%
      \put(951,409){\makebox(0,0){\strut{} 2.5}}%
      \put(1694,409){\makebox(0,0){\strut{} 2.55}}%
      \put(2436,409){\makebox(0,0){\strut{} 2.6}}%
      \put(3179,409){\makebox(0,0){\strut{} 2.65}}%
      \put(3922,409){\makebox(0,0){\strut{} 2.7}}%
      \put(4665,409){\makebox(0,0){\strut{} 2.75}}%
      \put(5407,409){\makebox(0,0){\strut{} 2.8}}%
      \put(6150,409){\makebox(0,0){\strut{} 2.85}}%
      \put(6893,409){\makebox(0,0){\strut{} 2.9}}%
      \csname LTb\endcsname%
      \put(144,2178){\rotatebox{-270}{\makebox(0,0){\strut{}f070w-f090w}}}%
      \csname LTb\endcsname%
      \put(3922,130){\makebox(0,0){\strut{}f090w-f115w}}%
      \put(3922,4040){\makebox(0,0){\strut{}Redshift 7.5--8.5}}%
    }%
    \gplgaddtomacro\gplfronttext{%
    }%
    \gplbacktext
    \put(0,0){\includegraphics{GRAPH_color_graph2}}%
    \gplfronttext
  \end{picture}%
\endgroup

						}\endgroup
				\caption{B\label{fig:col2}}
			\end{minipage}
			\begin{minipage}[c]{0.5\linewidth}
				\centering
					\begingroup\endlinechar=-1
						\resizebox{\textwidth}{!}{%
							% GNUPLOT: LaTeX picture with Postscript
\begingroup
  \makeatletter
  \providecommand\color[2][]{%
    \GenericError{(gnuplot) \space\space\space\@spaces}{%
      Package color not loaded in conjunction with
      terminal option `colourtext'%
    }{See the gnuplot documentation for explanation.%
    }{Either use 'blacktext' in gnuplot or load the package
      color.sty in LaTeX.}%
    \renewcommand\color[2][]{}%
  }%
  \providecommand\includegraphics[2][]{%
    \GenericError{(gnuplot) \space\space\space\@spaces}{%
      Package graphicx or graphics not loaded%
    }{See the gnuplot documentation for explanation.%
    }{The gnuplot epslatex terminal needs graphicx.sty or graphics.sty.}%
    \renewcommand\includegraphics[2][]{}%
  }%
  \providecommand\rotatebox[2]{#2}%
  \@ifundefined{ifGPcolor}{%
    \newif\ifGPcolor
    \GPcolortrue
  }{}%
  \@ifundefined{ifGPblacktext}{%
    \newif\ifGPblacktext
    \GPblacktexttrue
  }{}%
  % define a \g@addto@macro without @ in the name:
  \let\gplgaddtomacro\g@addto@macro
  % define empty templates for all commands taking text:
  \gdef\gplbacktext{}%
  \gdef\gplfronttext{}%
  \makeatother
  \ifGPblacktext
    % no textcolor at all
    \def\colorrgb#1{}%
    \def\colorgray#1{}%
  \else
    % gray or color?
    \ifGPcolor
      \def\colorrgb#1{\color[rgb]{#1}}%
      \def\colorgray#1{\color[gray]{#1}}%
      \expandafter\def\csname LTw\endcsname{\color{white}}%
      \expandafter\def\csname LTb\endcsname{\color{black}}%
      \expandafter\def\csname LTa\endcsname{\color{black}}%
      \expandafter\def\csname LT0\endcsname{\color[rgb]{1,0,0}}%
      \expandafter\def\csname LT1\endcsname{\color[rgb]{0,1,0}}%
      \expandafter\def\csname LT2\endcsname{\color[rgb]{0,0,1}}%
      \expandafter\def\csname LT3\endcsname{\color[rgb]{1,0,1}}%
      \expandafter\def\csname LT4\endcsname{\color[rgb]{0,1,1}}%
      \expandafter\def\csname LT5\endcsname{\color[rgb]{1,1,0}}%
      \expandafter\def\csname LT6\endcsname{\color[rgb]{0,0,0}}%
      \expandafter\def\csname LT7\endcsname{\color[rgb]{1,0.3,0}}%
      \expandafter\def\csname LT8\endcsname{\color[rgb]{0.5,0.5,0.5}}%
    \else
      % gray
      \def\colorrgb#1{\color{black}}%
      \def\colorgray#1{\color[gray]{#1}}%
      \expandafter\def\csname LTw\endcsname{\color{white}}%
      \expandafter\def\csname LTb\endcsname{\color{black}}%
      \expandafter\def\csname LTa\endcsname{\color{black}}%
      \expandafter\def\csname LT0\endcsname{\color{black}}%
      \expandafter\def\csname LT1\endcsname{\color{black}}%
      \expandafter\def\csname LT2\endcsname{\color{black}}%
      \expandafter\def\csname LT3\endcsname{\color{black}}%
      \expandafter\def\csname LT4\endcsname{\color{black}}%
      \expandafter\def\csname LT5\endcsname{\color{black}}%
      \expandafter\def\csname LT6\endcsname{\color{black}}%
      \expandafter\def\csname LT7\endcsname{\color{black}}%
      \expandafter\def\csname LT8\endcsname{\color{black}}%
    \fi
  \fi
  \setlength{\unitlength}{0.0500bp}%
  \begin{picture}(7200.00,4320.00)%
    \gplgaddtomacro\gplbacktext{%
      \put(849,595){\makebox(0,0)[r]{\strut{} 11}}%
      \put(849,912){\makebox(0,0)[r]{\strut{} 11.5}}%
      \put(849,1228){\makebox(0,0)[r]{\strut{} 12}}%
      \put(849,1545){\makebox(0,0)[r]{\strut{} 12.5}}%
      \put(849,1861){\makebox(0,0)[r]{\strut{} 13}}%
      \put(849,2178){\makebox(0,0)[r]{\strut{} 13.5}}%
      \put(849,2495){\makebox(0,0)[r]{\strut{} 14}}%
      \put(849,2811){\makebox(0,0)[r]{\strut{} 14.5}}%
      \put(849,3128){\makebox(0,0)[r]{\strut{} 15}}%
      \put(849,3444){\makebox(0,0)[r]{\strut{} 15.5}}%
      \put(849,3761){\makebox(0,0)[r]{\strut{} 16}}%
      \put(951,409){\makebox(0,0){\strut{} 1.5}}%
      \put(1694,409){\makebox(0,0){\strut{} 2}}%
      \put(2437,409){\makebox(0,0){\strut{} 2.5}}%
      \put(3179,409){\makebox(0,0){\strut{} 3}}%
      \put(3922,409){\makebox(0,0){\strut{} 3.5}}%
      \put(4665,409){\makebox(0,0){\strut{} 4}}%
      \put(5408,409){\makebox(0,0){\strut{} 4.5}}%
      \put(6150,409){\makebox(0,0){\strut{} 5}}%
      \put(6893,409){\makebox(0,0){\strut{} 5.5}}%
      \csname LTb\endcsname%
      \put(144,2178){\rotatebox{-270}{\makebox(0,0){\strut{}Y-J}}}%
      \csname LTb\endcsname%
      \put(3922,130){\makebox(0,0){\strut{}J-H}}%
      \put(3922,4040){\makebox(0,0){\strut{}Redshift 8.5--10.1}}%
    }%
    \gplgaddtomacro\gplfronttext{%
    }%
    \gplbacktext
    \put(0,0){\includegraphics{GRAPH_color_graph3}}%
    \gplfronttext
  \end{picture}%
\endgroup

						}\endgroup
				\caption{C\label{fig:col3}}
			\end{minipage}
			\begin{minipage}[c]{0.5\linewidth}
				\centering
					\begingroup\endlinechar=-1
						\resizebox{\textwidth}{!}{%
							% GNUPLOT: LaTeX picture with Postscript
\begingroup
  \makeatletter
  \providecommand\color[2][]{%
    \GenericError{(gnuplot) \space\space\space\@spaces}{%
      Package color not loaded in conjunction with
      terminal option `colourtext'%
    }{See the gnuplot documentation for explanation.%
    }{Either use 'blacktext' in gnuplot or load the package
      color.sty in LaTeX.}%
    \renewcommand\color[2][]{}%
  }%
  \providecommand\includegraphics[2][]{%
    \GenericError{(gnuplot) \space\space\space\@spaces}{%
      Package graphicx or graphics not loaded%
    }{See the gnuplot documentation for explanation.%
    }{The gnuplot epslatex terminal needs graphicx.sty or graphics.sty.}%
    \renewcommand\includegraphics[2][]{}%
  }%
  \providecommand\rotatebox[2]{#2}%
  \@ifundefined{ifGPcolor}{%
    \newif\ifGPcolor
    \GPcolortrue
  }{}%
  \@ifundefined{ifGPblacktext}{%
    \newif\ifGPblacktext
    \GPblacktexttrue
  }{}%
  % define a \g@addto@macro without @ in the name:
  \let\gplgaddtomacro\g@addto@macro
  % define empty templates for all commands taking text:
  \gdef\gplbacktext{}%
  \gdef\gplfronttext{}%
  \makeatother
  \ifGPblacktext
    % no textcolor at all
    \def\colorrgb#1{}%
    \def\colorgray#1{}%
  \else
    % gray or color?
    \ifGPcolor
      \def\colorrgb#1{\color[rgb]{#1}}%
      \def\colorgray#1{\color[gray]{#1}}%
      \expandafter\def\csname LTw\endcsname{\color{white}}%
      \expandafter\def\csname LTb\endcsname{\color{black}}%
      \expandafter\def\csname LTa\endcsname{\color{black}}%
      \expandafter\def\csname LT0\endcsname{\color[rgb]{1,0,0}}%
      \expandafter\def\csname LT1\endcsname{\color[rgb]{0,1,0}}%
      \expandafter\def\csname LT2\endcsname{\color[rgb]{0,0,1}}%
      \expandafter\def\csname LT3\endcsname{\color[rgb]{1,0,1}}%
      \expandafter\def\csname LT4\endcsname{\color[rgb]{0,1,1}}%
      \expandafter\def\csname LT5\endcsname{\color[rgb]{1,1,0}}%
      \expandafter\def\csname LT6\endcsname{\color[rgb]{0,0,0}}%
      \expandafter\def\csname LT7\endcsname{\color[rgb]{1,0.3,0}}%
      \expandafter\def\csname LT8\endcsname{\color[rgb]{0.5,0.5,0.5}}%
    \else
      % gray
      \def\colorrgb#1{\color{black}}%
      \def\colorgray#1{\color[gray]{#1}}%
      \expandafter\def\csname LTw\endcsname{\color{white}}%
      \expandafter\def\csname LTb\endcsname{\color{black}}%
      \expandafter\def\csname LTa\endcsname{\color{black}}%
      \expandafter\def\csname LT0\endcsname{\color{black}}%
      \expandafter\def\csname LT1\endcsname{\color{black}}%
      \expandafter\def\csname LT2\endcsname{\color{black}}%
      \expandafter\def\csname LT3\endcsname{\color{black}}%
      \expandafter\def\csname LT4\endcsname{\color{black}}%
      \expandafter\def\csname LT5\endcsname{\color{black}}%
      \expandafter\def\csname LT6\endcsname{\color{black}}%
      \expandafter\def\csname LT7\endcsname{\color{black}}%
      \expandafter\def\csname LT8\endcsname{\color{black}}%
    \fi
  \fi
  \setlength{\unitlength}{0.0500bp}%
  \begin{picture}(7200.00,4320.00)%
    \gplgaddtomacro\gplbacktext{%
      \put(645,595){\makebox(0,0)[r]{\strut{} 10}}%
      \put(645,947){\makebox(0,0)[r]{\strut{} 15}}%
      \put(645,1299){\makebox(0,0)[r]{\strut{} 20}}%
      \put(645,1650){\makebox(0,0)[r]{\strut{} 25}}%
      \put(645,2002){\makebox(0,0)[r]{\strut{} 30}}%
      \put(645,2354){\makebox(0,0)[r]{\strut{} 35}}%
      \put(645,2706){\makebox(0,0)[r]{\strut{} 40}}%
      \put(645,3057){\makebox(0,0)[r]{\strut{} 45}}%
      \put(645,3409){\makebox(0,0)[r]{\strut{} 50}}%
      \put(645,3761){\makebox(0,0)[r]{\strut{} 55}}%
      \put(747,409){\makebox(0,0){\strut{} 0}}%
      \put(1515,409){\makebox(0,0){\strut{} 2}}%
      \put(2284,409){\makebox(0,0){\strut{} 4}}%
      \put(3052,409){\makebox(0,0){\strut{} 6}}%
      \put(3820,409){\makebox(0,0){\strut{} 8}}%
      \put(4588,409){\makebox(0,0){\strut{} 10}}%
      \put(5357,409){\makebox(0,0){\strut{} 12}}%
      \put(6125,409){\makebox(0,0){\strut{} 14}}%
      \put(6893,409){\makebox(0,0){\strut{} 16}}%
      \csname LTb\endcsname%
      \put(144,2178){\rotatebox{-270}{\makebox(0,0){\strut{}f150w-f200w}}}%
      \csname LTb\endcsname%
      \put(3820,130){\makebox(0,0){\strut{}f115w-f150w}}%
      \put(3820,4040){\makebox(0,0){\strut{}Redshift 10--14}}%
    }%
    \gplgaddtomacro\gplfronttext{%
    }%
    \gplbacktext
    \put(0,0){\includegraphics{GRAPH_color_graph4}}%
    \gplfronttext
  \end{picture}%
\endgroup

						}\endgroup
				\caption{D\label{fig:col4}}
			\end{minipage}
			\caption{Graphs showing the colour-colour regions for \\
			(A) z=6-7.5. The colour window was defined as $f070w-f090w{\ge}3.257$ and $f090w-f115w{\le}8.242$,\\
			(B) z=7.5-8.5. The colour window was defined as $f090w-f115w{\ge}8.251$ and $f115w-f150w{\le}2.869$, \\
			(C) z=8.54-10.1. The colour window was defined as $Y-J{\ge}11.019$ and $J-H{\le}5.298$, \\
			(D) z=10-14. The colour window was defined as $f115w-f150w{\ge}14.439$ and $f150w-f200w{\le}14.815$}
		\end{figure}
	%subsection Results_for_Colour (end)

	\subsection{Interpretation of Colour Results}
	\label{sub:Interp_Colour}
		Constraining a colour window for observations turned out to be a challenging task. The filter 275w on James Webb didn't work properly in Hyperz; some magnitudes came out as 99.0 which is the default answer if there was an error; other magnitudes were above 100. As the source of this error could not be found, the range of redshift 14 to 15 was taken from the colour window results. The colour window results  shown in Figures~\ref{fig:col1},~\ref{fig:col2},~\ref{fig:col3} and~\ref{fig:col4} are much higher than expected results such as the colour windows in Lorenzoni et al. \cite{lorenzoni2013constraining}. Something must be wrong with the technique that was used to find the colours, however in defense of the results, the assumptions made as described in the sections above were reasonably fair for the timescale of the project. It's hard to say if there is one set of inputs or parameters that were the main source of error.  The shape of the distribtution of the galaxies is odd but seems to follow a pattern for each redshift range, which remains unexplained. The other peculiar characteristic of the diagrams is that all the galaxies appear on a mainly continous line, as opposed to being distributed in a certain area. Having done some basic tests, the cause of this appears to be due to setting the age of the galaxies at the same redshift to be the same; as soon as the age was randomised to some time between the formation redshift and the observed redshift, the galaxies were more spread out. However these galaxies appeared `inside' the colour window, so this doesn't appear to have affected the determination of the colour windows.
	%section Interpetation_of_colour_results (end)
%section Photometry_and_colour (end)
