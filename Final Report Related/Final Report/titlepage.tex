%!TEX root = mainfile.tex
\begin{titlepage}
  \begin{center}
    \vspace*{\fill}

    \centering
    \includegraphics[scale=1.0]{Logo.pdf}
    \vfill

    \hrule
    {\LARGE\bf Extragalactic Astrophysics and Cosmology\\Cosmic Reionization \\[0.4cm]}
    \hrule

    \vfill
    \large
    School of Physics and Astronomy\\
    University of Birmingham

    \vfill
    { Joe Baumber,
    	James Bryant,
    	Lewis Clegg,
    	Bethany Johnson,
    	Andrew King,
    	Owen McConnell,
    	Catherine McDonald,
    	Michael O'Neill,
    	Jonathan Shepley,
    	Dorothy Stonell,
    	Rahim Topadar,
    	Josh Wainwright\\}
    \vfill

    \vfill
    \textit{Supervisors:} Graham Smith, Alistair Sanderson, Melissa Gillone \\
    		\vfill
    \textit{Date:} March 2013
    \vfill
    \vfill

    \begin{abstract}
        This study deduces that reionization began at a redshift of $z=17.82(+3.06, -2.4)$ and ended at a redshift of $z=7\pm 1.8$. This is calculated by directly applying the dynamics of star formation and the ionization rate of neutral hydrogen in the Inter-galactic Medium. A photometry strategy consisting of 3 multi-band surveys is proposed in order to observe Lyman Break Galaxies across redshifts 6--17. The surveys will locate $100.5\pm37.0$, $138.7\pm 100.6$, $358.1\pm 158.6$ galaxies in redshift ranges 6--8.5, 8.5--10 and 10--17 respectively. These surveys will be completed by the James Webb Space Telescope and Euclid which are planned for launch in the coming decade. A follow up spectroscopy survey will be used to confirm the redshift and properties of 24, 4 and 48 galaxies in these 3 surveys respectively. The spectroscopy will be carried out using James Webb Space Telescope and a combination of single and multi-slit spectroscopy. It is shown that the use of known gravitational lenses, located between redshift 0.5--0.7, is very beneficial for discovering high redshift candidates as it can increase the depth of surveys by up to 3 magnitudes.
    \end{abstract}


  \end{center}
\end{titlepage}

%\thispagestyle{empty}
%\vspace*{\fill}
%\noindent
%\begin{tabular}{ll}
%\end{tabular}

%\cleardoublepage
%\cleardoublepage
