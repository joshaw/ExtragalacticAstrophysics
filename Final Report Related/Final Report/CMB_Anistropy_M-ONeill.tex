%!TEX root = mainfile.tex

\subsection{CMB Polarization and Anisotropies} % (fold)
\label{sub:cmb_secondary_anisotropies}
(Mike)

	Over large angular scales, the CMB polarization contains key information concerning the evolution of ionization during the EoR. The CMB photons released during recombination experienced considerable Thomson scattering off of free electrons. This scattering is imprinted onto the CMB anisotropy map introducing secondary anisotropies. This has the effect of removing anisotropies on smaller scales and introducing polarization anisotropies. By comparing the anisotropies observed with models of the CMB without reionization it is possible to determine the electron column density during the EoR. Using this method it is possible to calculate the period over which reionization took place. It is also possible to make measurements of the metal enrichment history of the IGM at the EoR. On-resonance scattering off metals and the influence of inverse Compton scattering (the Sunyaev-Zel'dovich effect) introduced additional signals into the CMB\cite{Monteagudo2006}. Studies in this field are expected to get a new impetus in the coming years with the results from the Planck telescope giving greater insight into the CMB map.
% subsection cmb_secondary_anisotropies (end)
