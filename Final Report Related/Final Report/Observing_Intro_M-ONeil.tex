%!TEX root = mainfile.tex
\section{Observing Strategy Group} % (fold)
\label{sec:observing_strategy_group}
(Mike)

	This primary aim of this subgroup is to formulate an observing strategy capable of probing the depths of the EoR. Our strategy is going to be based upon using multi-band photometry to detect candidate Lyman Break Galaxies (LBG) and confirming their properties using spectroscopy.

	The study of this era in the universe's history has come a long way in the past 30 years and with the many new telescopes and arrays being designed currently it is only set to accelerate over the coming decades. It is an understatement to say such distant redshifts are very difficult to see and it is a testament to scientific and engineering achievement that obtaining detailed images such as the ones currently being produced is possible. The light from these galaxies is so faint that it can take a very long time to see anything. Due to this long project duration, time on telescopes is in high demand.

	The strategy must therefore be as complete as possible with as many influencing factors considered. This strategy will focus on using the most efficient methods available in order to limit the observing time required. The second focus will be on utilising future telescopes to observe a much greater number of galaxies than previous surveys and to make observations above redshift 10, where little has been seen, in order to break through new frontiers and observe what happened at the earliest moments of structure formation. Our ambitious strategy will look to utilise the capabilities of the new technology to further the scientific understanding of the EoR.

	Our strategy will look to utilise the capabilities of the new technology to further the scientific understanding of the EoR.

	The observing strategy will be established as follows:
	\begin{itemize}
		\item Research possible methods for studying the full range of reionization.
		\item Explore the advantages and disadvantages of ground and space-based telescopes.
		\item Identify the most efficient telescope for a low redshift survey covering redshifts of 6-10.
		\item Identify the most efficient telescope for a high redshift survey covering redshifts of $>10$.
		\item Research gravitational lensing and its application in assisting our surveys to see to greater depths.
		\item Identify a telescope capable of spectroscopically confirming the nature and redshift of the candidate galaxies.
		\item Investigate colour; the use of colour-colour diagrams is essential for removing contaminants from the samples.
		\item Compile a final strategy, using the predictions from the predictions subgroup, capable of studying LBGs and constraining the limits of the EoR.
	\end{itemize}
% section observing_strategy_group (end)
